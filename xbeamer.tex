\iffalse meta-comment

File: xbeamer.tex Copyright (C) 2023-2025 Joseph Wright

It may be distributed and/or modified under the conditions of the
LaTeX Project Public License (LPPL), either version 1.3c of this
license or (at your option) any later version.  The latest version
of this license is in the file

   https://www.latex-project.org/lppl.txt

This file is part of the "xbeamer bundle" (The Work in LPPL)
and all files in that bundle must be distributed together.

The released version of this bundle is available from CTAN.

-----------------------------------------------------------------------

The development version of the bundle can be found at

   https://github.com/josephwright/xbeamer

for those people who are interested.

-----------------------------------------------------------------------

\fi

\DocumentMetadata{pdfversion = 2.0, lang = en}

\documentclass{l3doc}

\usepackage{siunitx}

% Commands needed in this document
\ExplSyntaxOn
\makeatletter
\NewDocumentCommand \acro { m }
  {
    \textsc
      {
        \exp_args:NV \tl_if_head_eq_charcode:nNTF \f@series { m }
          { \text_lowercase:n }
          { \use:n }
            {#1}
      }
  }
\makeatother
\ExplSyntaxOff
\NewDocumentCommand\email{m}{\href{mailto:#1}{\nolinkurl{#1}}}
\NewDocumentCommand\foreign{m}{\textit{#1}}
\NewDocumentCommand\opt{m}{\texttt{#1}}
\ExplSyntaxOn
\NewDocumentCommand \sarg { m } { \texttt < \__codedoc_meta:n {#1} \texttt > }
\ExplSyntaxOff
% Tidy up the above in bookmarks
\makeatletter
\pdfstringdefDisableCommands{%
  \let\acro\@firstofone
  \let\foreign\@firstofone
  \let\opt\@firstofone
}
\makeatother

% As we are dealing with a class, this has to be done manually
\def\filedate{2025-03-19}
\def\fileversion{v0.0.0}

\begin{document}

\title{%
  \cls{xbeamer} -- A class for typesetting presentations%
  \thanks{This file describes \fileversion,
    last revised \filedate.}%
}

\author{%
  Joseph Wright%
  \thanks{%
    E-mail: \email{joseph@texdev.net}%
  }%
}

\date{Released \filedate}

\maketitle

\tableofcontents

\begin{documentation}

\section{Introduction%
  \label{sec:intro}}

The \cls{beamer} class was first released in 2003, and rapidly became the
most popular method for producing presentations in \LaTeX{}. As detailed in
the \cls{beamer} manual
\begin{quotation}
  Till Tantau created \cls{beamer} mainly in his spare time. Many other people
  have helped by sending him emails containing suggestions for improvement or
  corrections or patches or whole new themes (by now, this amounts to over a
  thousand emails concerning \cls{beamer}). Indeed, most of the development was
  only initiated by feature requests and bug reports. Without this feedback,
  \cls{beamer} would still be what it was originally intended to be: a small
  private collection of macros that make using the \pkg{seminar} class easier.
  Till created the first version of \cls{beamer} for his PhD defense
  presentation in February 2003. A month later, he put the package on
  \acro{ctan} at the request of some colleagues. After that, things somehow got
  out of hand.
\end{quotation}

The effort which Till put in over the time he was developing \cls{beamer}
cannot be underestimated. However, there are several challenges which confront
us today
\begin{enumerate}
  \item The document interface is flexible but in places deviates from normal
    \LaTeX{} conventions
  \item Internally, the code makes use of whatever methods would give the
    visual results but not necessarily the most idiomatic style
  \item Till made few comments in the code or in commit messages in the code
    history
  \item The underlying box structure of a \cls{beamer} document is very
    different from the standard \LaTeX{} model, and a lot of material is
    boxed up multiple times
  \item Engine, font and macro development over the past 20 years offers
    new approaches for some areas
\end{enumerate}
These all feed into an issue addressing a major requirement today:
accessibility. Broadly, the internal structure (and to some extend the user
interface) of \cls{beamer} mean that it is not possible to \enquote{retrofit}
PDF tagging into the class.

Instead, the approach is to develop a new class, currently called
\cls{xbeamer}, which takes ideas from \cls{beamer} but with tagging and
accessibility of structure as a design aim from the beginning. In contrast to
work by the \LaTeX{} Project Team on making the core classes accessible, the
expectation for \cls{xbeamer} is that users \emph{will} need to change their
sources. Unlike other documents, presentations tend to be \enquote{single use}:
revised and adjusted each time they are used. The need to edit sources
should therefore not be an insurmountable barrier.

At present, this code is \emph{highly} experimental: only a (small) subset of
\cls{beamer} functionality is implemented, some things are being done
differently, almost everything is still subject to discussion.

\section{Simple example documents}

Currently, \cls{xbeamer} \emph{absolutely requires} the use of the
\cs{DocumentMetadata} command \emph{and} the a setting for the \texttt{tagging}
key. As such, the most basic \cls{xbeamer} document is
\begin{verbatim}
  \DocumentMetadata{tagging = on} % or "tagging = draft"
  \documentclass{xbeamer}
  \begin{document}
  \begin{frame}
    Some content here
  \end{frame}
  \end{document}
\end{verbatim}

A slightly more useful version, which generates multiple slides and shows some
(current) features, is
\begin{verbatim}
  \DocumentMetadata{tagging = on}
  \documentclass{xbeamer}
  \begin{document}
  \begin{frame}
    \frametitle{An example frame}
    \begin{itemize}[action-spec = <+->]
      \item This will be on slide one onward
      \item This will be on slide two onward
      \item<.-> So will this
      \item But this will only appear on slide three
    \end{itemize}
    Back to appearing on all slides
  \end{frame}
  \end{document}
\end{verbatim}

Tagging is activated for the standard (projector) output of \cls{xbeamer}, but
it is more useful in handout mode, which is activated using a class option.
\begin{verbatim}
  \DocumentMetadata{tagging = on}
  \documentclass[mode = handout]{xbeamer}
  \begin{document}
  \begin{frame}
    \frametitle{An example frame}
    \begin{itemize}[action-spec = <+->]
      \item This will be on slide one onward
      \item This will be on slide two onward
      \item<.-> So will this
      \item But this will only appear on slide three
    \end{itemize}
    Back to appearing on all slides
  \end{frame}
  \end{document}
\end{verbatim}

A larger set of examples which can be edited and typeset in the browser are
available at \url{https://www.texdev.net/xbeamer/examples/}.

\section{Class structure and design decisions}

As covered in Section~\ref{sec:intro}, the \cls{xbeamer} class is currently
highly experimental. Active discussion is ongoing around many aspects of the
way that things should work, and very little is therefore at all stable. That
said, some decisions have been made: some of this is re-stating ideas which
carry forward from \cls{beamer}.
\begin{itemize}
  \item The basic structure of a presentation is made up of \env{frame}
    environments, which are made up of one or more slides.
  \item Variable content is indicated by an \emph{overlay specification}, given
    in optional angle brackets (|< ... >|).
  \item Slides have a \emph{fixed} height of \qty{100}{\mm}.
  \item The default font will be sanserif using established standard
    implementations (currently \pkg{sansmathfonts} for \pdfTeX{} and New
    Computer Modern for OpenType engines).
\end{itemize}

\section{Differences from \cls{beamer}}

The following key differences between \cls{beamer} and \cls{xbeamer} are
important to note for the user
\begin{itemize}
  \item The default font setup in \cls{xbeamer} is \emph{all} sanserif, in
    particular \cs{mathrm} and \cs{textrm} will give a sanserif font
  \item Overlay specifications can only be given as the first argument to
    commands
  \item There are no optional braced arguments in \cls{xbeamer}, in particular
    frame titles are given using \cs{frametitle} and not as
    |\begin{frame}{...}|
  \item New overlay-aware commands and environments should be defined using
    \pkg{ltcmd} (\cs{NewDocumentCommand} and so on): no changes are made to the
    behavior of \cs{newcommand} or \cs{newenvironment}
  \item Where a command takes a star as an option, this comes \emph{before}
    the overlay argument, e.g. |\includgraphics*<...>|: this reflects the
    fact that the star is usually described as part of the command name rather
    than as an argument
\end{itemize}

\section{Creating frames}

\subsection{The frame environment}

\DescribeEnv{frame}
A presentation consists of a series of frames. Each frame consists of a series
of slides. You create a frame using \env{frame} environment. All of the text
that is not tagged by overlay specifications is shown on all slides of the
frame. (Overlay specifications are explained in more detail in later sections.
For the moment, let's just say that an overlay specification is a list of
numbers or number ranges in angle brackets that is put after certain commands
as in |\uncover<1,2>{Text}|.) If a frame contains commands that have an overlay
specification, the frame will contain multiple slides; otherwise it contains
only one slide.

\begin{verbatim}
  \begin{frame}
    \frametitle{A title}
    Some content
  \end{frame}
\end{verbatim}

\subsection{Frame and margin size}

\subsection{Restricting the slides of a frame}

The number of slides in a frame is automatically calculated. If the largest
number mentioned in any overlay specification inside the frame is~4, four
slides are introduced (despite the fact that a specification like |<4->| might
suggest that more than four slides would be possible).

\section{Creating overlays}

\subsection{The \cs{pause} command}

\DescribeMacro{\pause}
The \cs{pause} command offers an easy, but not very flexible way of creating
frames that are uncovered piecewise. If you say \cs{pause} somewhere in a
frame, only the text on the frame up to the \cs{pause} command is shown on the
first slide. On the second slide, everything is shown up to the second
\cs{pause}, and so forth. You can also use \cs{pause} inside environments; its
effect will last after the environment. However, taking this to extremes and
using \cs{pause} deeply within a nested environment may not have the desired
result.

A much more fine-grained control over what is shown on each slide can be
attained using overlay specifications, see the next sections. However, for many
simple cases the \cs{pause} command is sufficient.

The \cs{pause} command takes an optional argument
\begin{syntax}
  \cs{pause}\oarg{overlay spec}
\end{syntax}
This causes the text following it to be shown only from the next slide on, or,
if the optional \meta{overlay spec} is given, from the slide with the number
\meta{overlay spec}. If the optional \meta{overlay spec} is given, the counter
|pauses| is set to this number. This command uses the \cs{onslide} command
internally. The effect of \cs{pause} lasts till the next \cs{pause},
\cs{onslide}, or the end of the frame.
\begin{verbatim}
  \begin{frame}
    \begin{itemize}
    \item
      Shown from first slide on.
    \pause
    \item
      Shown from second slide on.
      \begin{itemize}
      \item
        Shown from second slide on.
      \pause
      \item
        Shown from third slide on.
      \end{itemize}
    \item
      Shown from third slide on.
    \pause
    \item
      Shown from fourth slide on.
    \end{itemize}

    Shown from fourth slide on.

    \begin{itemize}
    \onslide
    \item
      Shown from first slide on.
    \pause
    \item
      Shown from fifth slide on.
    \end{itemize}
  \end{frame}
\end{verbatim}

This command does \emph{not} work correctly inside math mode environments like
\env{align} and \pkg{pgf} environments like \env{tikzpicture} or
\env{tcolorbox}, since these do really wicked things.

To \enquote{unpause} some text, that is, to temporarily suspend pausing, use
the command \cs{onslide}, see below.

\subsection{The general concept of overlay specifications}

Whilst the \cs{pause} command is easy to understand, it is quite limited and so
is best suited only to simple cases. The \cls{xbeamer} class therefore supports
a different approach. The idea is to add \emph{overlay specifications} to
commands. These specifications are always given in pointed brackets and follow
the command as the first argument. In the simplest case, the specification
contains just a number. A command with an overlay specification following it
will only have \enquote{effect} on the slide(s) mentioned in the specification.
What exactly \enquote{having an effect} means, depends on the command. Consider
the following example.
\begin{verbatim}
  \begin{frame}
    \textbf{This line is bold on all three slides.}
    \textbf<2>{This line is bold only on the second slide.}
    \textbf<3>{This line is bold only on the third slide.}
  \end{frame}
\end{verbatim}

For the command \cs{textbf}, the overlay specification causes the text to be
set in boldface only on the specified slides. On all other slides, the text is
set in a normal font.

For a second example, consider the following frame:
\begin{verbatim}
  \begin{frame}
    \only<1>{This line is inserted only on slide 1.}
    \only<2>{This line is inserted only on slide 2.}
  \end{frame}
\end{verbatim}

The command \cs{only} normally simply inserts its parameter into the current
frame. However, if an overlay specification is present, it \enquote{throws
away} its parameter on slides that are not mentioned.

Overlay specifications can only be written behind certain commands, not every
command. Which commands you can use and which effects this will have is
explained in the next section. However, it is quite easy to redefine an
existing command such that it becomes \enquote{overlay specification aware},
see also Section~\ref{sec:overlay-cmds}.

The syntax of (basic) overlay specifications is the following: they are
comma-separated lists of slides and ranges. Ranges are specified like this:
|2-5|, which means slide two through to five. The start or the end of a range
can be omitted. For example, |3-| means \enquote{slides three, four, five, and
so on} and |-5| means the same as |1-5|. A complicated example is
|-3,6-8,10,12-15|, which selects the slides 1, 2, 3, 6, 7, 8, 10, 12, 13, 14
and~15.

\subsection{Commands with overlay specifications%
  \label{sec:overlay-cmds}}

For the following commands, adding an overlay specification causes the command
to be simply ignored on slides that are not included in the specification:
\cs{textbf}, \cs{textit}, \cs{textmd}, \cs{textnormal}, \cs{textrm},
\cs{textsc}, \cs{textsf}, \cs{textsl}, \cs{texttt}, \cs{textup}, \cs{emph};
\cs{color}, \cs{textcolor}; \cs{alert}, \cs{structure}. If a command takes
several arguments, like \cs{color}, the specification should directly follow
the command as in the following example:
\begin{verbatim}
  \begin{frame}
    \color<2-3>[rgb]{1,0,0} This text is red on slides 2 and 3, otherwise
      black.
  \end{frame}
\end{verbatim}

For the following commands, the effect of an overlay specification is special:

\DescribeMacro{\onslide}
\begin{syntax}
  \cs{onslide}\sarg{overlay spec}
\end{syntax}
All text following this command will only be shown (uncovered) on the specified slides. 
On non-specified slides, the text still occupies space. If no slides are specified, the 
following text is always shown. You need not call this command in the same \TeX{} group, 
its effect transcends block groups.
\begin{verbatim}
  \begin{frame}
    Shown on first slide.
    \onslide<2-3>
    Shown on second and third slide.
    \begin{itemize}
    \item
      Still shown on the second and third slide.
    \onslide<4->
    \item
      Shown from slide 4 on.
    \end{itemize}
    Shown from slide 4 on.
    \onslide
    Shown on all slides.
  \end{frame}
\end{verbatim}

\DescribeMacro{\only}
\begin{syntax}
  \cs{only}\sarg{overlay spec}\marg{text}
\end{syntax}
The \meta{text} is inserted only into the specified slides. For other slides, 
the text is simply thrown away. In particular, it occupies no space.
\begin{verbatim}
  \only<3->{Text inserted from slide 3 on.}
\end{verbatim}

\DescribeMacro{\uncover}
\begin{syntax}
  \cs{uncover}\sarg{overlay spec}\marg{text}
\end{syntax}
If the \meta{overlay spec} is present, the \meta{text} is shown
(\enquote{uncovered}) only on the specified slides. On other slides, the text
still occupies space and it is still typeset, but it is not shown or only shown
as if transparent. For details on how to specify whether the text is invisible
or just transparent see Section~\ref{sec:opacity}.
\begin{verbatim}
  \uncover<3->{Text shown from slide 3 on.}
\end{verbatim}

\DescribeMacro{\visible}
\begin{syntax}
  \cs{visible}\sarg{overlay spec}\marg{text}
\end{syntax}
This command does almost the same as \cs{uncover}. The only difference is that
if the text is not shown, it is never shown in a transparent way, but rather it
is not shown at all. Thus, for this command the transparency settings have no
effect.
\begin{verbatim}
  \visible<2->{Text shown from slide 2 on.}
\end{verbatim}

\DescribeMacro{\invisible}
\begin{syntax}
  \cs{invisible}\sarg{overlay spec}\marg{text}
\end{syntax}
This command does the opposite of \cs{visible}.
\begin{verbatim}
  \invisible<2->{Text hidden from slide 2 on.}
\end{verbatim}

\DescribeMacro{\alt}
\begin{syntax}
  \cs{alt}\sarg{overlay spec}\marg{default text}\marg{alternative text}
\end{syntax}
The default text is shown on the specified slides, otherwise the alternative
text.
\begin{verbatim}
  \alt<2>{On Slide 2}{Not on slide 2.}
\end{verbatim}

\DescribeMacro{\temporal}
\begin{syntax}
  \cs{temporal}\sarg{overlay spec}\marg{before slide text}\marg{default text}%
    \marg{after slide text}
\end{syntax}
This command alternates between three different texts, depending on whether the
current slide is temporally before the specified slides, is one of the
specified slides, or comes after them. If the \meta{overlay spec} is not an
interval (that is, if it has a \enquote{hole}), the \enquote{hole} is
considered to be part of the before slides.
\begin{verbatim}
  \temporal<3-4>{Shown on 1, 2}{Shown on 3, 4}{Shown 5, 6, 7, ...}
  \temporal<3,5>{Shown on 1, 2, 4}{Shown on 3, 5}{Shown 6, 7, 8, ...}
\end{verbatim}
As a possible application of the \cs{temporal} command consider the following example:
\begin{verbatim}
  \NewDocumentCommand\colorize{D<>{all}{%
    \temporal<#1>{\color{red!50}}{\color{black}}{\color{black!50}}}
  \begin{frame}
    \begin{itemize}
      \colorize<1> \item First item.
      \colorize<2> \item Second item.
      \colorize<3> \item Third item.
      \colorize<4> \item Fourth item.
    \end{itemize}
  \end{frame}
\end{verbatim}

\DescribeMacro{\item}
\begin{syntax}
  \cs{item}\sarg{action spec}\oarg{item label}
\end{syntax}
The effect of \meta{action spec} is described in Section~\ref{sec:action-spec};
this extends the \meta{overlay spec} to include the potential to
\enquote{alert} items.
\begin{verbatim}
  \begin{frame}
    \begin{itemize}
    \item<1-> First point, shown on all slides.
    \item<2-> Second point, shown on slide 2 and later.
    \item<2-> Third point, also shown on slide 2 and later.
    \item<3-> Fourth point, shown on slide 3.
    \end{itemize}
  \end{frame}

  \begin{frame}
    \begin{enumerate}
    \item<3-| alert@3>[0.] A zeroth point, shown at the very end.
    \item<1-| alert@1> The first and main point.
    \item<2-| alert@2> The second point.
    \end{enumerate}
  \end{frame}
\end{verbatim}

\subsection{Environments with overlay specifications}

\DescribeEnv{onlyenv}
\DescribeEnv{invisibleenv}
\DescribeEnv{uncoverenv}
\DescribeEnv{visibleenv}
For each of the basic commands \cs{only}, \cs{visible}, \cs{uncover} and
\cs{invisible} there exists \enquote{environment versions} \env{onlyenv},
\env{visibleenv}, \env{uncoverenv} and \env{invisibleenv}. Except for
\env{onlyenv}, these environments do the same as the commands.

For the \env{onlyenv} environment, the contents of the environment is inserted
into the text only on the specified slides. The difference to \cs{only} is,
that the text is actually typeset inside a box that is then thrown away,
whereas \cs{only} immediately throws away its contents. If the text is not
\enquote{typesettable}, the \env{onlyenv} may produce an error where \cs{only}
would not.
\begin{verbatim}
  \begin{frame}
    This line is always shown.
    \begin{onlyenv}<2>
      This line is inserted on slide 2.
    \end{onlyenv}
  \end{frame}
\end{verbatim}

\subsection{Dynamically changing text or images}

You may sometimes wish to have some part of a frame change dynamically from
slide to slide. On each slide of the frame, something different should be shown
inside this area. You could achieve the effect of dynamically changing text by
giving a list of \cs{only} commands like this:
\begin{verbatim}
  \only<1>{Initial text.}
  \only<2>{Replaced by this on second slide.}
  \only<3>{Replaced again by this on third slide.}
\end{verbatim}

The trouble with this approach is that it may lead to slight, but annoying
differences in the heights of the lines, which may cause the whole frame to
\enquote{wobble} from slide to slide. This problem becomes much more severe if
the replacement text is several lines long.

To solve this problem, you can use two environments: \env{overlayarea} and
\env{overprint}. The first is more flexible, but less user-friendly.

\DescribeEnv{overlayarea}
\begin{syntax}
  \cs{begin}\{overlayarea\}\marg{area width}\marg{area height}
\end{syntax}
Everything within the environment will be placed in a rectangular area of the
specified size. The area will have the same size on all slides of a frame,
regardless of its actual contents.
\begin{verbatim}
  \begin{overlayarea}{\textwidth}{3cm}
    \only<1>{Some text for the first slide.\\Possibly several lines long.}
    \only<2>{Replacement on the second slide.}
  \end{overlayarea}
\end{verbatim}

\DescribeEnv{overprint}
\begin{syntax}
  \cs{begin}\{overprint\}\oarg{area width}
\end{syntax}
The \meta{area width} defaults to the text width. Inside the environment, use
\cs{only} or \cs{onslide} to specify different things that should be shown for
this environment on different slides. Everything within the environment will be
placed in a rectangular area of the specified width. The height and depth of
the area are chosen large enough to accommodate the largest contents of the
area. Two compilations will be needed to allow \LaTeX{} to track the tallest
version of the frame contents.
\begin{verbatim}
  \begin{overprint}
    \only<1|handout:1>{%
        Some text for the first slide.\\
        Possibly several lines long.
      }%
    \only<2|handout:0>{%
       Replacement on the second slide. Suppressed for handout.
     }%
  \end{overprint}
  Following text
\end{verbatim}

A similar need for dynamical changes arises when you have, say, a series of
pictures named \file{first}, \file{second}, and \file{third} that show
different stages of some process. To make a frame that shows these pictures on
different slides, the following code might be used:
\begin{verbatim}
  \begin{frame}
    \frametitle{The Three Process Stages}

    \includegraphics<1>{first}
    \includegraphics<2>{second}
    \includegraphics<3>{third}
  \end{frame}
\end{verbatim}

\section{Structuring a presentation: the local structure}

\LaTeX{} provides different commands for structuring text \enquote{locally},
for example, \foreign{via} the \env{itemize} environment. These environments
are also available in the \cls{xbeamer} class, although their appearance has
been slightly changed.

\subsection{Itemizations, enumerations and descriptions}

\DescribeEnv{description}
\DescribeEnv{enumerate}
\DescribeEnv{itemize}
There are three predefined environments for creating lists, namely
\env{enumerate}, \env{itemize}, and \env{description}. The first two can be
nested to depth three, but nesting them to this depth creates totally
unreadable slides.

\DescribeMacro{\item}
The \cs{item} command is overlay specification-aware. If an overlay
specification is provided, the item will only be shown on the specified slides,
see the following example. If the \cs{item} command is to take an optional
argument and an overlay specification, the overlay specification comes first as
in |\item<1>[Cat]|.
\begin{verbatim}
  \begin{frame}
    There are three important points:
    \begin{enumerate}
      \item<1-> A first one,
      \item<2-> a second one with a bunch of subpoints,
        \begin{itemize}
          \item first subpoint. (Only shown from second slide on!).
          \item<3-> second subpoint added on third slide.
          \item<4-> third subpoint added on fourth slide.
        \end{itemize}
      \item<5-> and a third one.
    \end{enumerate}
  \end{frame}
\end{verbatim}

The list environments have syntax
\begin{syntax}
  \cs{begin}\marg{list type}\oarg{options}
\end{syntax}

If the option \opt{action-spec} is given, in every occurrence of an \cs{item}
command that does not have an overlay specification attached to it, the
\meta{overlay specification} is used. By setting this specification to be an
incremental overlay specification, see Section~\ref{XXX}, you
can implement, for example, a stepwise uncovering of the items. The
\meta{overlay specification} is inherited by subenvironments.
Naturally, in a subenvironment you can reset it locally by setting it to
|<1->| (the subitems will be shown on all slides) or |<.->| (the subitems will
be shown starting from the same slide as the parent item).
\begin{verbatim}
  \begin{itemize}[action-spec = <+->]
    \item This is shown from the first slide on.
    \item This is shown from the second slide on.
    \item This is shown from the third slide on.
    \item<1-> This is shown from the first slide on.
    \item This is shown from the fourth slide on.
  \end{itemize}
\end{verbatim}
If you do not need to give any other options for the list environment, you may
use the shortened format |[<...>]|, which matches the \cls{beamer} syntax:
\begin{verbatim}
  \begin{itemize}[<+->]
    \item This is shown from the first slide on.
    \item This is shown from the second slide on.
    \item This is shown from the third slide on.
    \item<1-> This is shown from the first slide on.
    \item This is shown from the fourth slide on.
  \end{itemize}
\end{verbatim}

\subsection{Highlighting}

\subsection{Block environments}

\subsection{Splitting a frame into multiple columns}

\end{documentation}

\PrintIndex

\end{document}
