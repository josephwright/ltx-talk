% \iffalse meta-comment
%
% File: xbeamer-overlay.dtx Copyright (C) 2024,2025 Joseph Wright
%
% It may be distributed and/or modified under the conditions of the
% LaTeX Project Public License (LPPL), either version 1.3c of this
% license or (at your option) any later version.  The latest version
% of this license is in the file
%
%    https://www.latex-project.org/lppl.txt
%
% This file is part of the "xbeamer bundle" (The Work in LPPL)
% and all files in that bundle must be distributed together.
%
% The released version of this bundle is available from CTAN.
%
% -----------------------------------------------------------------------
%
% The development version of the bundle can be found at
%
%    https://github.com/josephwright/xbeamer
%
% for those people who are interested.
%
% -----------------------------------------------------------------------
%
%<*driver>
\documentclass{l3doc}
% Additional commands needed in this source
\NewDocumentCommand\email{m}{\href{mailto:#1}{\nolinkurl{#1}}}
\begin{document}
  \DocInput{\jobname.dtx}
\end{document}
%</driver>
% \fi
%
% ^^A As we are dealing with a class, this has to be done manually
% \def\filedate{2025-04-27}
% \def\fileversion{v0.0.0}
%
% \title{^^A
%   \pkg{xbeamer-overlay} -- Overlays^^A
%   \thanks{This file describes \fileversion,
%     last revised \filedate.}^^A
% }
%
% \author{^^A
%  Joseph Wright^^A
%  \thanks{^^A
%    E-mail: \email{joseph@texdev.net}^^A
%   }^^A
% }
%
% \date{Released \filedate}
%
% \maketitle
%
% \begin{documentation}
%
% \end{documentation}
%
% \begin{implementation}
%
% \section{\pkg{xbeamer-overlay} implementation}
%
% Start the \pkg{DocStrip} guards.
%    \begin{macrocode}
%<*class>
%    \end{macrocode}
%
% Identify the internal prefix.
%    \begin{macrocode}
%<@@=xbeamer>
%    \end{macrocode}
%
% \subsection{Utilities}
%
% \begin{macro}[TF]{\@@_if_overlay:n, \@@_if_overlay:V}
% \begin{macro}{\@@_overlay_arg:n}
%    \begin{macrocode}
\prg_new_protected_conditional:Npnn \@@_if_overlay:n #1 { T , F , TF }
  {
    \@@_decode_parse:n {#1}
    \bool_if:NTF \l_@@_decode_overlays_bool
      \prg_return_true:
      \prg_return_false:
  }
\prg_generate_conditional_variant:Nnn \@@_if_overlay:n { V } { T , F , TF }
%    \end{macrocode}
%   A macro processor variant of the check that always results in an |N|-type
%   bool.
%    \begin{macrocode}
\cs_new_protected:Npn \@@_overlay_arg:n #1
  {
    \@@_if_overlay:nTF {#1}
      { \cs_set:Npn \ProcessedArgument { \c_true_bool } }
      { \cs_set:Npn \ProcessedArgument { \c_false_bool } }
  }
%    \end{macrocode}
% \end{macro}
% \end{macro}
%
% \subsection{Action commands and environments}
%
% Commands that can be used as actions all have a common form (with one
% exception). The common internal structure is used to enable them to be used
% as actions by looking for the name
% \cs[no-index]{@@_action_\meta{name}_begin:}. There are two forms: opt-in and
% opt-out: to deal with that we use named conditionals.
%
% \begin{macro}
%   {
%     \@@_if_action_alert:Nn     ,
%     \@@_if_action_visible:Nn   ,
%     \@@_if_action_only:Nn      ,
%     \@@_if_action_uncover:Nn   ,
%     \@@_if_action_invisible:Nn
%   }
%   To allow for flipping logic, the payload argument here may be opt-in or
%   opt-out.
%    \begin{macrocode}
\cs_new_eq:NN \@@_if_action_alert:Nn \bool_if:NT
\cs_new_eq:NN \@@_if_action_invisible:Nn \bool_if:NT
\cs_new_eq:NN \@@_if_action_only:Nn \bool_if:NF
\cs_new_eq:NN \@@_if_action_uncover:Nn \bool_if:NF
\cs_new_eq:NN \@@_if_action_visible:Nn \bool_if:NF
%    \end{macrocode}
% \end{macro}
% \begin{macro}{\@@_if_action_:Nn}
%   We also provide a version to cover the case there is no action at
%   all: this always skips the payload.
%    \begin{macrocode}
\cs_new_eq:NN \@@_if_action_:Nn \use_none:nn
%    \end{macrocode}
% \end{macro}
%
% \begin{macro}{\@@_action_alert_begin:, \@@_action_alert_end:}
%   Currently a rather simple approach!
%    \begin{macrocode}
\cs_new_protected:Npn \@@_action_alert_begin:
  {
    \group_begin:
      \color_select:n { alert }
  }
\cs_new_protected:Npn \@@_action_alert_end:
  { \group_end: }
%    \end{macrocode}
% \end{macro}
%
% \begin{macro}
%   {
%     \@@_action_invisible_begin:, \@@_action_invisible_end:,
%     \@@_action_visible_begin:,   \@@_action_visible_end:
%   }
%   Simply hide unconditionally.
%    \begin{macrocode}
\cs_new_protected:Npn \@@_action_invisible_begin:
  {
    \group_begin:
      \opacity_select:n { 0 }
  }
\cs_new_protected:Npn \@@_action_invisible_end:
  { \group_end: }
\cs_new_eq:NN \@@_action_visible_begin: \@@_action_invisible_begin:
\cs_new_eq:NN \@@_action_visible_end: \@@_action_invisible_end:
%    \end{macrocode}
% \end{macro}
%
% \begin{macro}{\@@_action_only_begin:, \@@_action_only_end:}
%   Here, we simply throw away the content we do not want: this is done by
%   typesetting in a disposable box.
%    \begin{macrocode}
\cs_new_protected:Npn \@@_action_only_begin:
  { \vbox_set:Nw \l_@@_tmp_box }
\cs_new_protected:Npn \@@_action_only_end:
  { \vbox_set_end: }
%    \end{macrocode}
% \end{macro}
%
% \begin{variable}{\l_@@_uncover_hidden_fp}
%   Currently just an on-off, but that will change.
%    \begin{macrocode}
\NewTemplateType { hidden } { 0 }
\DeclareTemplateInterface { hidden } { xbeamer } { 0 }
  { opacity : real = 0 }
\DeclareTemplateCode { hidden } { xbeamer } { 0 }
  { opacity = \l_@@_uncover_hidden_fp }
  { \opacity_select:n { \l_@@_uncover_hidden_fp } }
\DeclareInstance { hidden } { std } { xbeamer } { }
%    \end{macrocode}
% \end{variable}
% \begin{macro}{\@@_action_uncover_begin:, \@@_action_uncover_end:}
%   Use the template
%    \begin{macrocode}
\cs_new_protected:Npn \@@_action_uncover_begin:
  {
    \group_begin:
      \UseInstance { hidden } { std }
  }
\cs_new_protected:Npn \@@_action_uncover_end:
  { \group_end: }
%    \end{macrocode}
% \end{macro}
%
% \begin{macro}{\only, \invisible, \uncover}
%   Commands and environments where the payload applies when the material is
%   not active on the slide.
%    \begin{macrocode}
\clist_map_inline:nn { only , invisible , uncover }
  {
    \ExpandArgs { cne } \NewDocumentCommand {#1} { D <> { all } +m }
      {
        \exp_not:N \@@_if_overlay:nTF {##1}
          {##2}
          {
            \exp_not:c { @@_action_ #1 _begin: }
            ##2
            \exp_not:c { @@_action_ #1 _end: }
          }
      }
%    \end{macrocode}
% \end{macro}
% \begin{environment}{onlyenv, invisibleenv, uncoverenv}
%    \begin{macrocode}
    \ExpandArgs { nnee } \NewDocumentEnvironment { #1 env }
      { > { \@@_overlay_arg:n } D <> { all } }
      {
        \exp_not:N \bool_if:NF ##1
          { \exp_not:c { @@_action_ #1 _begin: } }
      }
      {
        \exp_not:N \bool_if:NF ##1
          { \exp_not:c { @@_action_ #1 _end: } }
      }
  }
%    \end{macrocode}
% \end{environment}
% \begin{macro}{\alert, \visible}
%   And those where the action applies when we are on the slide.
%    \begin{macrocode}
\clist_map_inline:nn { alert , visible }
  {
    \ExpandArgs { cne } \NewDocumentCommand {#1} { D <> { all } +m }
      {
        \exp_not:N \@@_if_overlay:nTF {##1}
          {
            \exp_not:c { @@_action_ #1 _begin: }
            ##2
            \exp_not:c { @@_action_ #1 _end: }
          }
          {##2}
      }
%    \end{macrocode}
% \end{macro}
% \begin{environment}{alertenv, visibleenv}
%    \begin{macrocode}
    \ExpandArgs { nnee } \NewDocumentEnvironment { #1 env }
      { > { \@@_overlay_arg:n } D <> { all } }
      {
        \exp_not:N \bool_if:NF ##1
          { \exp_not:c { @@_action_ #1 _begin: } }
      }
      {
        \exp_not:N \bool_if:NF ##1
          { \exp_not:c { @@_action_ #1 _end: } }
      }
  }
%    \end{macrocode}
% \end{environment}
% \begin{macro}{\only}
%   This code needs to be done manually as for the command version the content
%   must be entirely discarded. That can't work for the environment version,
%   which has to deal with for example single items in a list (and so cannot
%   be collected up verbatim and must use a box).
%    \begin{macrocode}
\RenewDocumentCommand \only { D <> { all } +m }
  {
    \@@_if_overlay:nT {#1}
      {#2}
  }
%    \end{macrocode}
% \end{macro}
%
% \begin{variable}
%   {
%     \l_@@_saved_overlays_bool ,
%     \l_@@_saved_action_str    ,
%     \l_@@_saved_actions_bool
%   }
%    \begin{macrocode}
\bool_new:N \l_@@_saved_overlays_bool
\str_new:N \l_@@_saved_action_str
\bool_new:N \l_@@_saved_actions_bool
%    \end{macrocode}
% \end{variable}
%
% \begin{macro}{\action}
% \begin{environment}{actionenv}
% \begin{macro}{\@@_action_begin:n}
% \begin{macro}{\@@_action_end:}
%   As we need data on not just overlays but also actions at the end of the
%   environment, this has to be done manually. To allow working with
%   environments but also items, the code needs to save data for the end
%   function. The group is needed for cases where we are not in a \LaTeX{}
%   environment group.
%    \begin{macrocode}
\NewDocumentCommand \action { D <> { all } +m }
  {
    \@@_action_begin:n {#1}
    #2
    \@@_action_end:
  }
\NewDocumentEnvironment { actionenv } { D <> { all } }
  { \@@_action_begin:n {#1} }
  { \@@_action_end: }
\cs_new_protected:Npn \@@_action_begin:n #1
  {
    \group_begin:
      \@@_decode_parse:n {#1}
      \bool_set_eq:NN \l_@@_saved_overlays_bool
        \l_@@_decode_overlays_bool
      \str_set_eq:NN \l_@@_saved_action_str
        \l_@@_decode_action_str
      \bool_set_eq:NN \l_@@_saved_actions_bool
        \l_@@_decode_actions_bool
      \bool_if:NTF \l_@@_decode_overlays_bool
        {
          \use:c { @@_if_action_ \l_@@_decode_action_str :Nn }
            \l_@@_decode_actions_bool
            { \use:c { @@_action_ \l_@@_decode_action_str _begin: } }
        }
        { \@@_action_uncover_begin: }
  }
\cs_new_protected:Npn \@@_action_end:
  {
      \bool_if:NTF \l_@@_saved_overlays_bool
        {
          \use:c { @@_if_action_ \l_@@_saved_action_str :Nn }
            \l_@@_saved_actions_bool
            { \use:c { @@_action_ \l_@@_saved_action_str _end: } }
        }
        { \@@_action_uncover_end: }
    \group_end:
  }
%    \end{macrocode}
% \end{macro}
% \end{macro}
% \end{environment}
% \end{macro}
%
% \subsection{Non-action commands and environments}
%
% This section contains commands and environments that do \emph{not}
% need to be made available as actions.
%
% \begin{macro}{\alt}
%   Simple wrappers around the internal switch.
%    \begin{macrocode}
\NewDocumentCommand \alt { D <> { all } +m +m }
  {
    \@@_if_overlay:nTF {#1}
      {#2}
      {#3}
  }
%    \end{macrocode}
% \end{macro}
%
% \begin{macro}{\onslide}
% \begin{macro}{\@@_onslide:n}
% \begin{macro}{\@@_onslide_reset:}
%   Simply make transparent: we will likely need to save the original
%   opacity level. To allow us to apply independent of group level, a little
%   work is needed.
%    \begin{macrocode}
\NewDocumentCommand \onslide { D <> { all } }
  { \@@_onslide:n {#1} }
\cs_new_protected:Npn \@@_onslide:n #1
  {
    \tl_use:N \g_@@_onslide_tl
    \@@_if_overlay:nTF {#1}
      { \@@_onslide_reset: }
      {
        \opacity_select:n { 0 }
        \tl_gset:Nn \g_@@_onslide_escape_tl
          {
            \opacity_select:n { 0 }
            \group_insert_after:N \g_@@_onslide_escape_tl
          }
        \group_insert_after:N \g_@@_onslide_escape_tl
        \tl_gset:Nn \g_@@_onslide_tl
          {
            \tl_gclear:N \g_@@_onslide_tl
            \tl_gclear:N \g_@@_onslide_escape_tl
            \@@_onslide_reset:
          }
      }
  }
\cs_new_protected:Npn \@@_onslide_reset: { \opacity_select:n { 1 } }
%    \end{macrocode}
% \end{macro}
% \end{macro}
% \end{macro}
% \begin{variable}{\g_@@_onslide_tl, \g_@@_onslide_escape_tl}
%    \begin{macrocode}
\tl_new:N \g_@@_onslide_tl
\tl_new:N \g_@@_onslide_escape_tl
%    \end{macrocode}
% \end{variable}
%
% \begin{macro}{\temporal}
%   A tricky one: to separate the not-on-current-slide cases, the flag to
%   continue is used.
%    \begin{macrocode}
\NewDocumentCommand \temporal { D <> { all } +m +m +m }
  {
    \@@_if_overlay:nTF {#1}
      {#3}
      {
        \bool_if:NTF \g_@@_slide_continue_bool
          {#4}
          {#2}
      }
  }
%    \end{macrocode}
% \end{macro}
%
% \begin{macro}{\pause}
%   A thin wrapper.
%    \begin{macrocode}
\NewDocumentCommand \pause { o }
  {
    \IfNoValueTF {#1}
      { \int_gincr:N \g_@@_pauses_int }
      { \int_gset:Nn \g_@@_pauses_int {#1} }
    \exp_args:Ne \@@_onslide:n { \int_use:N \g_@@_pauses_int - }
  }
%    \end{macrocode}
% \end{macro}
%
% \subsection{Fixed-size areas}
%
% \begin{macro}{\@@_overprint_begin:n}
%   A common auxiliary for overprinting, which starts off much the same for
%   both \env{overlayarea} and \env{overprint}.
%    \begin{macrocode}
\cs_new_protected:Npn \@@_overprint_begin:n #1
  {
    \par
    \vbox_set_to_wd:Nnw \l_@@_tmp_box {#1}
      \raggedright
      \ignorespaces
  }
%    \end{macrocode}
% \end{macro}
%
% \begin{environment}{overlayarea}
%   An initial approach: quite similar to a column.
%    \begin{macrocode}
\NewDocumentEnvironment { overlayarea } { m m }
  { \@@_overprint_begin:n {#1} }
  {
    \vbox_set_end:
    \vbox_to_ht:nn {#2}
      {
        \box_use_drop:N \l_@@_tmp_box
        \vfil
      }
    \par
  }
%    \end{macrocode}
% \end{environment}
%
% \begin{variable}{\l_@@_overprint_int}
%   Track the overprints on a slide: as the slide forms a group, we do not need
%   to worry about resetting.
%    \begin{macrocode}
\int_new:N \l_@@_overprint_int
%    \end{macrocode}
% \end{variable}
%
% \begin{macro}[EXP]{\@@_frame_overprint:}
%   To refer to the current overprint environment within the document:
%   needed in the \texttt{.aux} so avoids using non-letters.
%    \begin{macrocode}
\cs_new:Npn \@@_frame_overprint:
  {
    \int_to_Roman:n \g_@@_frame_int
    \int_to_roman:n \l_@@_overprint_int
  }
%    \end{macrocode}
% \end{macro}
%
% \begin{environment}{overprint}
% \begin{macro}{\@@_overprint_save_ht:}
% \begin{macro}{\@@_overprint_check_ht:n}
%   For overprinting, in contrast to \cls{beamer} we use a two-pass approach
%   to save the size at the end of the run: this means you can use \cs{only}
%   for example in overprinting.
%    \begin{macrocode}
\NewDocumentEnvironment { overprint } { O { \textwidth } }
  { \@@_overprint_begin:n {#1} }
  {
    \vbox_set_end:
    \int_incr:N \l_@@_overprint_int
    \@@_overprint_save_ht:
    \cs_if_exist:cTF 
      { overprint@ \@@_frame_overprint: }
      {
        \dim_compare:vNnTF
          { overprint@ \@@_frame_overprint: }
            > { \box_ht:N \l_@@_tmp_box }
          {
            \vbox_to_ht:vn
              { overprint@ \@@_frame_overprint: }
              {
                \box_use_drop:N \l_@@_tmp_box
                \vfil
              }
          }
          { \box_use_drop:N \l_@@_tmp_box }
      }
      { \box_use_drop:N \l_@@_tmp_box }
    \par
  }
%    \end{macrocode}
%   As there is no clear end-point for overprinting, we need to be careful to
%   keep the current width separate from the saved one. The rest is then about
%   saving to the \texttt{.aux} file and helping out the user. 
%    \begin{macrocode}
\cs_new_protected:Npn \@@_overprint_save_ht:
  {
    \tl_if_exist:cF { g_@@_overprint_ \@@_frame_overprint: _tl }
      {
        \tl_new:c { g_@@_overprint_ \@@_frame_overprint: _tl }
        \tl_gset:cn { g_@@_overprint_ \@@_frame_overprint: _tl }
          { 0pt }
      }
    \tl_gset:ce { g_@@_overprint_ \@@_frame_overprint: _tl }
      {
        \dim_max:vn { g_@@_overprint_ \@@_frame_overprint: _tl }
          { \box_ht:N \l_@@_tmp_box }
      }
    \legacy_if:nT { @filesw }
      {
        \iow_now:Ne \@auxout
          {
            \gdef \exp_not:c { overprint@ \@@_frame_overprint: }
              {
                \exp_not:v { g_@@_overprint_ \@@_frame_overprint: _tl }
              }
          }
      }
    \hook_gput_code:nne { enddocument / afterlastpage } { xbeamer }
       { \@@_overprint_check_ht:n { \@@_frame_overprint: } }
  }
\cs_new_protected:Npn \@@_overprint_check_ht:n #1
  {
    \bool_lazy_and:nnF
      { \exp_not:N \cs_if_exist_p:c { overprint@ #1 } }
      {
        \dim_compare_p:vNv { overprint@ #1 } = { g_@@_overprint_ #1 _tl }
      }
      {
        \msg_warning:nn { xbeamer } { overprint-ht }
        \cs_gset_protected:Npn \@@_overprint_check_ht:n ##1 { }
      }
  }
\msg_new:nnn { xbeamer } { overprint-ht }
  {
    Overprint~area~height~has~changed:\\
    rerun~LaTeX.
  }
%    \end{macrocode}
% \end{macro}
% \end{macro}
% \end{environment}
%
% \subsection{Adding overlays to existing commands}
%
% \begin{macro}
%  {
%    \textbf ,
%    \textit ,
%    \textmd ,
%    \textnormal ,
%    \textrm ,
%    \textsc ,
%    \textsf ,
%    \textsl ,
%    \texttt ,
%    \textup ,
%    \emph
%  }
% \begin{macro}
%  {
%    \stdtextbf ,
%    \stdtextit ,
%    \stdtextmd ,
%    \stdtextnormal ,
%    \stdtextrm ,
%    \stdtextsc ,
%    \stdtextsf ,
%    \stdtextsl ,
%    \stdtexttt ,
%    \stdtextup ,
%    \stdemph
%  }

% \begin{macro}{\@@_textcmd_eqiv:n}
%   Make the standard text commands overlay-aware. To keep the spacing
%   unchanged when the command is not active, we use the same approach as
%   the kernel does for inserting the right grouping.
%    \begin{macrocode}
\tl_map_inline:nn
  {
    \textbf
    \textit
    \textmd
    \textnormal
    \textrm
    \textsc
    \textsf
    \textsl
    \texttt
    \textup
    \emph
  }
  {
    \ExpandArgs { c } \NewCommandCopy { std \cs_to_str:N #1 } #1
    \ExpandArgs { Nne } \RenewDocumentCommand #1
      { D <> { all } +m }
      {
        \exp_not:N \@@_if_overlay:nTF {##1}
          { \exp_not:c { std \cs_to_str:N #1 } }
          { \exp_not:N \@@_textcmd_eqiv:n }
            {##2}
      }
  }
\cs_new_protected:Npn \@@_textcmd_eqiv:n #1
  {
    \mode_if_math:TF
      { { \mbox {#1} } }
      {
        \mode_leave_vertical:
        {#1}
      }
  }
%    \end{macrocode}
% \end{macro}
% \end{macro}
% \end{macro}
%
% \begin{macro}{\includegraphics}
% \begin{macro}{\stdincludegraphics}
%   Just wrap up the args and forward if appropriate. The star is |#1| here as
%   that matches the documented behavior of starred commands generally.
%    \begin{macrocode}
\RequirePackage { graphicx }
\NewCommandCopy \stdincludegraphics \includegraphics
\RenewDocumentCommand \includegraphics { s D <> { all } o o m }
  {
    \@@_if_overlay:nT {#2}
      {
        \use:e
          {
            \exp_not:N \stdincludegraphics
              \IfBooleanT #1 { * }
              \IfNoValueF {#3} { [ \exp_not:n { {#3} } ] }
              \IfNoValueF {#4} { [ \exp_not:n { {#4} } ] }
          }
            {#5}
      }
  }
%    \end{macrocode}
% \end{macro}
% \end{macro}
%
% \begin{macro}{\label}
% \begin{macro}{\@@_label:n}
%   Here, we can't wrap the existing command up as we need the space hack, so
%   it has to be declared from scratch. There is also a non-standard overlay
%   default. At present, no special tricks as seen in \pkg{beamer}.   
%    \begin{macrocode}
\RenewDocumentCommand \label { D <> { 1 } m }
  {
    \@bsphack
    \@@_if_overlay:nT {#1}
      { \@@_label:n {#2} }
    \@esphack
  }
\cs_new_protected:Npn \@@_label:n #1
  {
    \begingroup
      \UseHookWithArguments { label } { 1 } {#1}
      \protected@write \@auxout { }
        {
          \string \newlabel {#1}
            { 
              { \@currentlabel }
              { \thepage }
              { \@currentlabelname }
              { \@currentHref }
              { \@kernel@reserved@label@data }
            }
        }
    \endgroup
  }
%    \end{macrocode}
% \end{macro}
% \end{macro}
%
%    \begin{macrocode}
%</class>
%    \end{macrocode}
%
% \end{implementation}
%
% \PrintIndex
