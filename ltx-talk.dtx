% \iffalse meta-comment
%
% File: ltx-talk.dtx Copyright (C) 2023-2025 Joseph Wright
%
% It may be distributed and/or modified under the conditions of the
% LaTeX Project Public License (LPPL), either version 1.3c of this
% license or (at your option) any later version.  The latest version
% of this license is in the file
%
%    https://www.latex-project.org/lppl.txt
%
% This file is part of the "ltx-talk bundle" (The Work in LPPL)
% and all files in that bundle must be distributed together.
%
% The released version of this bundle is available from CTAN.
%
% -----------------------------------------------------------------------
%
% The development version of the bundle can be found at
%
%    https://github.com/josephwright/ltx-talk
%
% for those people who are interested.
%
% -----------------------------------------------------------------------
%
%<*driver>
\documentclass{l3doc}
% Additional commands needed in this source
\NewDocumentCommand\email{m}{\href{mailto:#1}{\nolinkurl{#1}}}
\begin{document}
  \DocInput{\jobname.dtx}
\end{document}
%</driver>
% \fi
%
% ^^A As we are dealing with a class, this has to be done manually
% \def\filedate{2025-07-14}
% \def\fileversion{v0.1.1}
%
% \title{^^A
%   \pkg{ltx-talk} -- Overall set up^^A
%   \thanks{This file describes \fileversion,
%     last revised \filedate.}^^A
% }
%
% \author{^^A
%  Joseph Wright^^A
%  \thanks{^^A
%    E-mail: \email{joseph@texdev.net}^^A
%   }^^A
% }
%
% \date{Released \filedate}
%
% \maketitle
%
% \begin{documentation}
%
% \end{documentation}
%
% \begin{implementation}
%
% \section{\pkg{ltx-talk} implementation}
%
% Start the \pkg{DocStrip} guards.
%    \begin{macrocode}
%<*class>
%    \end{macrocode}
%
% Identify the internal prefix.
%    \begin{macrocode}
%<@@=talk>
%    \end{macrocode}
%
% \subsection{Set up}
%
% Identify the package and give the over all version information.
%    \begin{macrocode}
\ProvidesExplClass {ltx-talk} {2025-07-14} {0.1.1}
  {A class for typesetting presentations}
%    \end{macrocode}
%
% Get the right type of message.
%    \begin{macrocode}
\prop_gput:Nnn \g_msg_module_name_prop { talk } { ltx-talk }
\prop_gput:Nnn \g_msg_module_type_prop { talk } { Class }
%    \end{macrocode}
%
% Require the latest \LaTeX{} structures.
%    \begin{macrocode}
\IfFormatAtLeastF{2025-11-01}
  {
    \msg_new:nnnn { ltx-talk } { kernel-too-old }
      { ltx-talk~ requires~ at~ least~ LaTeX2e~ 2025-11-01. }
      {
        You~ might~ need~ to~ compile~ using~ either~ lualatex-dev~ or~
        pdflatex-dev.
      }
    \msg_fatal:nn { ltx-talk } { kernel-too-old }
  }
\NeedsDocumentMetadata
%    \end{macrocode}
%
% \subsection{Additions for \pkg{expl3}}
%
% Like \cs{vcoffin_set:Nnn}, so should be an easy enough addition.
%    \begin{macrocode}
\cs_gset_protected:Npn \vbox_set_to_wd:Nnn #1#2#3
  {
    \tex_setbox:D #1 \tex_vbox:D
      {
        \tex_hsize:D \__box_dim_eval:n {#2}
        \color_group_begin: #3 \par \color_group_end:
      }
    \box_dp:N #1 \__box_dim_eval:n {#2}
  }
\cs_gset_protected:Npn \vbox_set_to_wd:Nnw #1#2
  {
    \cs_set_protected:Npn \__box_set_to_wd:
      { \box_wd:N #1 \__box_dim_eval:n {#2} }
    \tex_setbox:D #1 \tex_vbox:D
      \c_group_begin_token
        \tex_hsize:D \__box_dim_eval:n {#2}
        \group_insert_after:N \__box_set_to_wd:
        \color_group_begin:
  }
%    \end{macrocode}
%
% Some things from \pkg{xbox} that would be useful.
%    \begin{macrocode}
\cs_gset_protected:Npn \rule:nnn #1#2#3
  {
    \tex_vrule:D
      height \dim_eval:n {#2} \exp_stop_f:
      depth  \dim_eval:n {#3} \exp_stop_f:
      width  \dim_eval:n {#1} \exp_stop_f:
      \scan_stop:
  }
%    \end{macrocode}
%
% \subsection{Extra variants}
%
%    \begin{macrocode}
\cs_generate_variant:Nn \clist_set:Nn { cv }
\cs_generate_variant:Nn \hook_gput_code:nnn { nne }
\exp_args_generate:n { nVv }
\cs_generate_variant:Nn \color_select:n { V }
\cs_generate_variant:Nn \dim_compare:nNnTF { v }
\cs_generate_variant:Nn \dim_compare_p:nNn { vNv }
\cs_generate_variant:Nn \dim_max:nn { v }
\cs_generate_variant:Nn \text_purify:n { v }
\cs_generate_variant:Nn \vbox_to_ht:nn { v }
%    \end{macrocode}
%
% \subsection{Scratch space}
%
% \begin{macro}{\@@_tmp:w}
%   For one-off processing.
%    \begin{macrocode}
\cs_new_protected:Npn \@@_tmp:w { }
%    \end{macrocode}
% \end{macro}
%
% \begin{variable}{\l_@@_tmp_box}
%    \begin{macrocode}
\box_new:N \l_@@_tmp_box
%    \end{macrocode}
% \end{variable}
%
% \begin{variable}{\l_@@_tmp_tl}
%    \begin{macrocode}
\tl_new:N \l_@@_tmp_tl
%    \end{macrocode}
% \end{variable}
%
% \subsection{Option handling}
%
% \begin{variable}
%   {
%     \l_@@_aspect_ratio_str ,
%     \l_@@_fontsize_dim     ,
%     \l_@@_frame_title_bool ,
%     \l_@@_mode_str
%   }
%    \begin{macrocode}
\keys_define:nn { talk }
  {
    aspect-ratio .str_set:N =
      \l_@@_aspect_ratio_str ,
    font-size .dim_set:N =
      \l_@@_fontsize_dim ,
    frame-title-arg .bool_set:N =
      \l_@@_frame_title_bool ,
    mode .choices:nn =
      { handout , projector }
      { \str_set:NV \l_@@_mode_str \l_keys_choice_tl }
  }
%    \end{macrocode}
% \end{variable}
%
% Scope for options.
%    \begin{macrocode}
\keys_define:nn { talk }
  {
    aspect-ratio    .usage:n = load ,
    font-size       .usage:n = load ,
    frame-title-arg .usage:n = load ,
    mode            .usage:n = load
  }
%    \end{macrocode}
%
% Initial values.
%    \begin{macrocode}
\keys_set:nn { talk }
  {
    aspect-ratio    = 16:9      ,
    font-size       = 11pt      ,
    frame-title-arg = false     ,
    mode            = projector
  }
%    \end{macrocode}
%
%    \begin{macrocode}
\ProcessKeyOptions [ talk ]
%    \end{macrocode}
%
% \subsection{Setting up}
%
%  Load the font size setup if available, otherwise fall back on scaling.
%    \begin{macrocode}
\file_if_exist_input:nF { size \dim_to_decimal:n \l_@@_fontsize_dim .clo }
  {
    \file_input:n { size10.clo }
    \RequirePackage { relsize }
    \hook_gput_code:nne { begindocument } { talk }
      { \exp_not:N \relsize { \fp_eval:n { \l_@@_fontsize_dim / 10pt } } }
  }
%    \end{macrocode}
%
% \begin{variable}
%   {
%     \c_@@_paper_height_dim  ,
%     \c_@@_paper_width_dim
%   }
%   As \pkg{geometry} is being used to set the paper size with no previous
%   value, we have to use the optional argument rather than waiting to
%   apply \cs{geometry}.
%    \begin{macrocode}
\dim_const:Nn \c_@@_paper_height_dim { 100mm }
\use:e
  {
    \cs_set_protected:Npn \exp_not:N \@@_tmp:w
      #1 \tl_to_str:n { : } #2 \tl_to_str:n { : } #3 \exp_not:N \q_stop
      {
        \dim_const:Nn \exp_not:N \c_@@_paper_width_dim
          {
            \exp_not:N \fp_to_dim:n
              { (#1 / #2) * \exp_not:N \c_@@_paper_height_dim }
          }
      }
    \exp_not:N \@@_tmp:w \l_@@_aspect_ratio_str
      \tl_to_str:n { : } 100 \exp_not:N \q_stop
  }
\use:e
  {
    \exp_not:N \RequirePackage
      [
        papersize =
          {
            \dim_use:N \c_@@_paper_width_dim  ,
            \dim_use:N \c_@@_paper_height_dim
          } ,
        tmargin    = 10mm ,
        bmargin    =  8mm ,
        lmargin    = 10mm ,
        rmargin    = 10mm ,
        headheight = 10mm ,
        headsep    =  2mm ,
        footskip   =  6mm
      ]
      { geometry }
  }
%    \end{macrocode}
% \end{variable}
%
% Turn off justification
%    \begin{macrocode}
\raggedright
%    \end{macrocode}
%
% \subsection{Math support}
%
% We always require \pkg{amsmath}: this is forced anyway by \pkg{unicode-math}
% for Lua\TeX{}.
%    \begin{macrocode}
\RequirePackage { amsmath }
%    \end{macrocode}
%
% \subsection{Font selection}
%
% The aim here is to minimize change from the standard font setup but at the
% same time provide a sans-serif default. Since \cls{beamer} was released,
% better sans-serif math mode fonts have become available. For OpenType engines,
% requiring \pkg{unicode-math} is the most sensible approach. The New Computer
% Modern font provides a reasonable initial set of glyphs. It comes with a
% wrapper package, but that does various other things: if the user wants these,
% they can choose to load themselves. For 8-bit engines, switching the text
% font to be sans-serif is easy. For math mode, the \pkg{sansmathfonts} package
% does a good job: here, using the package rather than  adjusting directly is
% the sensible option.
%    \begin{macrocode}
\sys_if_engine_opentype:TF
  {
    \RequirePackage { unicode-math }
    \setsansfont { NewCMSans10-Regular.otf }
    \setmathfont { NewCMSansMath-Regular.otf }
  }
  { \RequirePackage { sansmathfonts } }
\cs_set_eq:NN \rmdefault \sfdefault
%    \end{macrocode}
%
% \subsection{Hyperlinks}
%
% \begin{variable}{\thepage}
%   We define \cs{thepage} here: this is checked for by \pkg{hyperref} so has
%   to come early.
%    \begin{macrocode}
\cs_new:Npn \thepage { \@arabic \c@page }
%    \end{macrocode}
% \end{variable}
%
% A requirement.
%    \begin{macrocode}
\RequirePackage { hyperref }
\hypersetup { hidelinks }
%    \end{macrocode}
%
% \subsection{Tagging}
%
% We need to extend the standard tagging model to work with slides and so on.
%    \begin{macrocode}
\tagpdfsetup
  {
    role / user-NS = ltx-talk        ,
    role / new-tag = frame / Sect    ,
    role / new-tag = frametitle / H4
  }
%    \end{macrocode}
%
%    \begin{macrocode}
%</class>
%    \end{macrocode}
%
% \end{implementation}
%
% \PrintIndex
