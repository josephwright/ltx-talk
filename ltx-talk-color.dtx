% \iffalse meta-comment
%
% File: ltx-talk-color.dtx Copyright (C) 2025 Joseph Wright
%
% It may be distributed and/or modified under the conditions of the
% LaTeX Project Public License (LPPL), either version 1.3c of this
% license or (at your option) any later version.  The latest version
% of this license is in the file
%
%    https://www.latex-project.org/lppl.txt
%
% This file is part of the "ltx-talk bundle" (The Work in LPPL)
% and all files in that bundle must be distributed together.
%
% The released version of this bundle is available from CTAN.
%
% -----------------------------------------------------------------------
%
% The development version of the bundle can be found at
%
%    https://github.com/josephwright/ltx-talk
%
% for those people who are interested.
%
% -----------------------------------------------------------------------
%
%<*driver>
\documentclass{l3doc}
% Additional commands needed in this source
\NewDocumentCommand\email{m}{\href{mailto:#1}{\nolinkurl{#1}}}
\begin{document}
  \DocInput{\jobname.dtx}
\end{document}
%</driver>
% \fi
%
% ^^A As we are dealing with a class, this has to be done manually
% \def\filedate{2025-07-19}
% \def\fileversion{v0.1.4}
%
% \title{^^A
%   \pkg{ltx-talk-color} -- Color definitions^^A
%   \thanks{This file describes \fileversion,
%     last revised \filedate.}^^A
% }
%
% \author{^^A
%  Joseph Wright^^A
%  \thanks{^^A
%    E-mail: \email{joseph@texdev.net}^^A
%   }^^A
% }
%
% \date{Released \filedate}
%
% \maketitle
%
% \begin{documentation}
%
% \end{documentation}
%
% \begin{implementation}
%
% \section{\pkg{ltx-talk-color} implementation}
%
% Start the \pkg{DocStrip} guards.
%    \begin{macrocode}
%<*class>
%    \end{macrocode}
%
% Identify the internal prefix.
%    \begin{macrocode}
%<@@=talk>
%    \end{macrocode}
%
% The aim here is to \emph{test} how well \pkg{l3color} can support the range
% of color functions that are needed for a presentation. As such, this is
% very much experimental, but deliberately so. In particular, there is an
% important question about the need for global colors: used throughout
% \cls{beamer} but otherwise not widely encountered. At the same time, there is
% a need to work with packages that expect color to be managed in a
% predictable way: \pkg{pgf} in particular makes use of \pkg{xcolor} internal
% as part of color management.
%
% Currently, colors defined using \pkg{xcolor} will be passed on to
% \pkg{l3color} provided \cs{DocumentMetadata} is active. As that is a
% requirement in any case for \cls{ltx-talk}, some of the setup is relatively
% easy to do.
%
% \subsection{Existing definitions}
%
%    \begin{macrocode}
\RequirePackage { xcolor }
%    \end{macrocode}
%
% \begin{macro}{\stdcolor, \stdmathcolor, \stdtextcolor}
%   Save the document commands.
%    \begin{macrocode}
\NewCommandCopy \stdcolor \color
\NewCommandCopy \stdmathcolor \mathcolor
\NewCommandCopy \stdtextcolor \textcolor
%    \end{macrocode}
% \end{macro}
%
% \subsection{Document commands}
%
%    \begin{macrocode}
\cs_generate_variant:Nn \color_select:n { e }
\cs_generate_variant:Nn \color_select:nn { ne }
\cs_generate_variant:Nn \color_math:nn { e }
\cs_generate_variant:Nn \color_math:nnn { ne }
%    \end{macrocode}
%
% \begin{macro}{\color, \mathcolor, \textcolor}
%   Add the overlay specification and use \pkg{l3color}.
%    \begin{macrocode}
\RenewDocumentCommand \color { D <> { all } o m }
  {
    \@@_if_overlay:nT {#1}
      {
        \IfNoValueTF {#2}
          { \color_select:e {#3} }
          { \color_select:ne {#2} {#3} }
      }
  }
\RenewDocumentCommand \mathcolor { D <> { all } o m +m }
  {
    \@@_if_overlay:nT {#1}
      {
        \IfNoValueTF {#2}
          { \color_math:en {#3} {#4} }
          { \color_math:nen {#2} {#3} {#4} }
      }
  }
\RenewDocumentCommand \textcolor { D <> { all } o m +m }
  {
    \@@_if_overlay:nT {#1}
      {
        \mode_leave_vertical:
        \group_begin:
          \IfNoValueTF {#2}
            { \color_select:e {#3} }
            { \color_select:ne {#2} {#3} }
          #4
        \group_end:
      }
  }
%    \end{macrocode}
% \end{macro}
%
% \subsection{Color definition}
%
% \begin{macro}{\DeclareColor}
%   Provide a single interface here: as the data will be passed to
%   \pkg{l3color} in any case, there is not too much to do.
%    \begin{macrocode}
\NewDocumentCommand \DeclareColor { m o m }
  {
    \IfNoValueTF {#2}
      { \colorlet {#1} {#3} }
      { \definecolor {#1} {#2} {#3} }
  }
%    \end{macrocode}
% \end{macro}
%
% \subsection{Semantic colors}
%
% Pick up the standard colors from \pkg{beamer}.
%    \begin{macrocode}
\DeclareColor { alert } [ RGB ] { 200 , 0 , 0 }
\DeclareColor { example } { green!50!black }
\DeclareColor { structure } [ rgb ] { 0.2 , 0.2 , 0.7 }
%    \end{macrocode}
%
%    \begin{macrocode}
%</class>
%    \end{macrocode}
%
% \end{implementation}
%
% \PrintIndex
