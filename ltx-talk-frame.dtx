% \iffalse meta-comment
%
% File: ltx-talk-frame.dtx Copyright (C) 2023-2025 Joseph Wright
%
% It may be distributed and/or modified under the conditions of the
% LaTeX Project Public License (LPPL), either version 1.3c of this
% license or (at your option) any later version.  The latest version
% of this license is in the file
%
%    https://www.latex-project.org/lppl.txt
%
% This file is part of the "ltx-talk bundle" (The Work in LPPL)
% and all files in that bundle must be distributed together.
%
% The released version of this bundle is available from CTAN.
%
% -----------------------------------------------------------------------
%
% The development version of the bundle can be found at
%
%    https://github.com/josephwright/ltx-talk
%
% for those people who are interested.
%
% -----------------------------------------------------------------------
%
%<*driver>
\documentclass{l3doc}
% Additional commands needed in this source
\NewDocumentCommand\email{m}{\href{mailto:#1}{\nolinkurl{#1}}}
\begin{document}
  \DocInput{\jobname.dtx}
\end{document}
%</driver>
% \fi
%
% ^^A As we are dealing with a class, this has to be done manually
% \def\filedate{2025-06-09}
% \def\fileversion{v0.0.0}
%
% \title{^^A
%   \pkg{ltx-talk-frame} -- The structure of frames^^A
%   \thanks{This file describes \fileversion,
%     last revised \filedate.}^^A
% }
%
% \author{^^A
%  Joseph Wright^^A
%  \thanks{^^A
%    E-mail: \email{joseph@texdev.net}^^A
%   }^^A
% }
%
% \date{Released \filedate}
%
% \maketitle
%
% \begin{documentation}
%
% \end{documentation}
%
% \begin{implementation}
%
% \section{\pkg{ltx-talk-frame} implementation}
%
% Start the \pkg{DocStrip} guards.
%    \begin{macrocode}
%<*class>
%    \end{macrocode}
%
% Identify the internal prefix.
%    \begin{macrocode}
%<@@=talk>
%    \end{macrocode}
%
% \subsection{Slides in frames}
%
% Currently, each slide in a frame will produce a separate page in the output
% (unless the slide is suppressed entirely). Material is then hidden on some
% pages by using opacity. An alternative approach would be to use Optional
% Content Groups to have a similar effect on one page per frame. However,
% whilst that would be relatively clear for appear/disappear effects, it would
% be much less straight-forward for partial transparency, \foreign{etc.}, plus
% would depend more heavily on viewer support. At a future stage we may wish to
% revisit this.
%
% \begin{variable}{\g_@@_slide_continue_bool}
%   Tracks whether the frame continues after the current slide.
%    \begin{macrocode}
\bool_new:N \g_@@_slide_continue_bool
%    \end{macrocode}
% \end{variable}
%
% \begin{variable}{\l_@@_slide_box}
%    \begin{macrocode}
\box_new:N \l_@@_slide_box
%    \end{macrocode}
% \end{variable}
%
% \begin{variable}{\g_@@_slide_int, \c@slide, \theslide}
%   The slide number inside the current frame: needed to know which overlays
%   are active. We also provide \LaTeX{} counter-style access.
%    \begin{macrocode}
\int_new:N \g_@@_slide_int
\cs_new_eq:NN \c@slide \g_@@_slide_int
\cs_new:Npn \theslide { \@arabic \c@slide }
%    \end{macrocode}
% \end{variable}
%
% Required to know which is the last slide in a frame for tagging.
%    \begin{macrocode}
\property_new:nnnn { slides } { now } { 1 } { \int_use:N \g_@@_slide_int }
%    \end{macrocode}
%
% \begin{macro}{\@@_slide:nn}
% \begin{macro}{\@@_slide_aux:n}
%  Each slide is parsed inside simple set up, the only complexity being if we
%  are handling fragile frames. There, all \cs{obeyedline} in the grabbed
%  tokens need to be turned back into |^^M| before rescanning: this ensures
%  that any verbatim grabbing in the frame still works. The strange business
%  with setting the continuation boolean is needed as otherwise we get an
%  infinite loop if there is an overlay specification for the frame itself.
%    \begin{macrocode}
\cs_new_protected:Npn \@@_slide:nn #1#2
  { 
    \group_begin:
      \tl_set:Ne \l_@@_tmp_tl
        {
          \property_ref:ee { frame . \int_use:N \g_@@_frame_int }
            { slides }
        }
      \str_if_eq:VnTF \l_@@_frame_tagging_str { n }
        { \str_set:NV  \l_@@_frame_tagging_str \l_@@_tmp_tl }
        {
          \str_replace_all:NnV \l_@@_frame_tagging_str { ,n }
            \l_@@_tmp_tl
          \str_replace_all:NnV \l_@@_frame_tagging_str { ,~n }
            \l_@@_tmp_tl
        }
      \int_gzero:N \g_@@_slide_int
      \RenewCommandCopy \frame \@@_latexe_frame:n
      \bool_do_while:Nn \g_@@_slide_continue_bool
        {
          \int_gincr:N \g_@@_slide_int
          \@@_if_overlay:nTF {#1}
            {
              \@@_slide_begin:
              \@@_if_overlay:VTF \l_@@_frame_tagging_str
                { \@@_frame_tag:n }
                { \@@_frame_notag:n }
                  {
                    \bool_if:NTF \l_@@_frame_verb_bool
                      { \@@_slide_aux:n }
                      { \use:n }
                        {#2}
                  }
               \@@_slide_end:
            }
            { \bool_gset_false:N \g_@@_slide_continue_bool }
        }
      \property_record:ee { frame . \int_use:N \g_@@_frame_int }
        { slides }
    \group_end:
  }
\cs_new_protected:Npn \@@_slide_aux:n #1
  {
    \group_begin:
      \cs_set:Npn \obeyedline { ^^J }
      \use:e
        {
          \group_end:
          \tl_retokenize:n {#1}
        }
  }
%    \end{macrocode}
% \end{macro}
% \end{macro}
%
% The very last frame will not be recorded by the above, so we have to add to
% the hook at the very end of the run.
%    \begin{macrocode}
\AddToHook { enddocument / afterlastpage }
  {
    \property_record:ee { frame . \int_use:N \g_@@_frame_int }
      { slides }
  }
%    \end{macrocode}
% 
% \begin{variable}{\c_@@_pause_init_int}
%   A simple concept: mainly done for performance reasons.
%    \begin{macrocode}
\cs_new_eq:NN \c_@@_pause_init_int \c_one_int
%    \end{macrocode}
% \end{variable}
%
% \begin{variable}{\g_@@_frame_struct_int}
%  The tagging structure number for the slide: needed by the content placed
%  outside of the current box (for example the frame title).
%    \begin{macrocode}
\int_new:N \g_@@_frame_struct_int
%    \end{macrocode}
% \end{variable}
%
% \begin{macro}{\@@_slide_begin:, \@@_slide_end:}
%    \begin{macrocode}
\cs_new_protected:Npn \@@_slide_begin:
  {
    \int_gset_eq:NN \g_@@_pauses_int \c_@@_pause_init_int
    \bool_gset_false:N \g_@@_slide_continue_bool
    \tl_gclear:N \g_@@_frame_title_tl
    \tl_gclear:N \g_@@_frame_subtitle_tl
    \@@_cnt_save:
    \vbox_set:Nw \l_@@_slide_box
      \tl_gclear:N \g_@@_onslide_tl
  }
\cs_new_protected:Npn \@@_slide_end:
  {
      \tl_use:N \g_@@_onslide_tl
    \vbox_set_end:
    \bool_if:NT \g_@@_slide_continue_bool
      { \@@_cnt_restore: }
    \vbox_to_ht:nn { \textheight }
      {
        \use:c { @@_slide_align_ \l_@@_frame_alignment_tl :n }
          { \vbox_unpack_drop:N \l_@@_slide_box }
      }
    \clearpage
  }
%    \end{macrocode}
% \end{macro}
%
% \begin{macro}
%   {
%     \@@_slide_align_bottom:n  ,
%     \@@_slide_align_center:n  ,
%     \@@_slide_align_stretch:n ,
%     \@@_slide_align_top:n
%   }
%   A pretty standard abstraction: we make sure there are always two skips.
%    \begin{macrocode}
\cs_new_protected:Npn \@@_slide_align_bottom:n #1
  {
    \skip_vertical:n { 0pt~plus~1fil }
    #1
    \skip_vertical:n { 0pt }
  }
\cs_new_protected:Npn \@@_slide_align_center:n #1
  {
    \skip_vertical:n { 0pt~plus~0.5fil }
    #1
    \skip_vertical:n { 0pt~plus~0.5fil }
  }
\cs_new_protected:Npn \@@_slide_align_stretch:n #1
  {
    \skip_vertical:n { 0pt }
    #1
    \skip_vertical:n { 0pt }
  }
\cs_new_protected:Npn \@@_slide_align_top:n #1
  {
    \skip_vertical:n { 0pt }
    #1
    \skip_vertical:n { 0pt~plus~1fil }
  }
%    \end{macrocode}
% \end{macro}
%
% \subsection{Counters}
%
% \begin{variable}{\l_@@_cnt_reset_seq}
%   As \cs{stepcounter}, \foreign{etc}., will increment at each overlay, there
%   is a need to save and reset. The list will be finalized at the end of the
%   preamble, so the data storage is created at that point. The starting point
%   is counters created before the class is loaded (other than those for lists,
%   which reset \enquote{naturally}). Other cases are handled by adding to
%   \cs{newcounter}.
%    \begin{macrocode}
\seq_new:N \l_@@_cnt_reset_seq
\seq_set_from_clist:Nn \l_@@_cnt_reset_seq
  {
    equation       ,
    footnote       ,
    mpfootnote     ,
    parentequation
  }
\seq_map_inline:Nn \l_@@_cnt_reset_seq
    {
      \int_new:c { g_@@_saved_ #1 _int }
      \int_gset_eq:cc { g_@@_saved_ #1 _int } { c@ #1 }
    }
%    \end{macrocode}
% \end{variable}
%
% \begin{macro}{\@@_cnt_save:, \@@_cnt_restore:}
%   A simple save-and-restore pair.
%    \begin{macrocode}
\cs_new_protected:Npn \@@_cnt_save:
  {
    \seq_map_inline:Nn \l_@@_cnt_reset_seq
      { \int_gset_eq:cc { g_@@_saved_ ##1 _int } { c@ ##1 } }
  }
\cs_new_protected:Npn \@@_cnt_restore:
  {
    \seq_map_inline:Nn \l_@@_cnt_reset_seq
      { \int_gset_eq:cc { c@ ##1 } { g_@@_saved_ ##1 _int } }
  }
%    \end{macrocode}
% \end{macro}
%
% \begin{macro}{\@definecounter, \std@definecounter}
%   Track all counters for resetting.
%    \begin{macrocode}
\cs_new_eq:NN \std@definecounter \@definecounter
\cs_gset_protected:Npn \@definecounter #1
  {
    \std@definecounter {#1}
    \int_new:c { g_@@_saved_ #1 _int }
    \seq_gput_right:Nn \l_@@_cnt_reset_seq {#1}
  }
%    \end{macrocode}
% \end{macro}
%
% \subsection{Frame options}
%
% \begin{variable}{\l_@@_frame_alignment_tl}
%    \begin{macrocode}
\tl_new:N \l_@@_frame_alignment_tl
%    \end{macrocode}
% \end{variable}
%
% \begin{variable}{\l_@@_action_spec_str, \l_@@_frame_tagging_str}
%    \begin{macrocode}
\keys_define:nn { talk / frame }
  {
    action-spec .str_set:N
      = \l_@@_action_spec_str ,
    tag-slides .str_set:N
      = \l_@@_frame_tagging_str ,
    vertical-alignment .choices:nn =
      { bottom , center , stretch , top }
      {
        \tl_set_eq:NN \l_@@_frame_alignment_tl
          \l_keys_value_tl
      }
  }
\keys_set:nn { talk / frame }
  {
    action-spec        =        ,
    tag-slides         = n      ,
    vertical-alignment = center
  }
%    \end{macrocode}
% \end{variable}
%
% \subsection{Tagging for headers}
%
% \begin{macro}{\@@_header_tag_begin:n, \@@_header_tag_begin:e}
% \begin{macro}{\@@_header_tag_end:}
%   Generalized control for inserting material into the header area (which is
%   otherwise outside of tagging).
%    \begin{macrocode}
\cs_new_protected:Npn \@@_header_tag_begin:n #1
  {
    \tag_resume:n { header }
    \tag_mc_end:
    \tag_struct_begin:n {#1}
    \tag_mc_begin:n { }
  }
\cs_generate_variant:Nn \@@_header_tag_begin:n { e }
\cs_new_protected:Npn \@@_header_tag_end:
  {
    \tag_mc_end:
    \tag_struct_end:
    \tag_mc_begin:n { artifact }
    \tag_suspend:n { header }
  }
%    \end{macrocode}
% \end{macro}
% \end{macro}
%
% \subsection{Wallpaper}
%
% \begin{variable}
%   {
%     \l_@@_footelem_left_skip ,
%     \l_@@_footelem_right_skip ,
%     \l_@@_footelem_color_tl ,
%     \l_@@_footelem_font_tl
%   }
%    \begin{macrocode}
\NewTemplateType { footer-element } { 1 }
\DeclareTemplateInterface { footer-element } { talk } { 1 }
  {
    color      : tokenlist ,
    font       : tokenlist = ,
    left-skip  : length = 0em ,
    right-skip : length = 0em
  }
\DeclareTemplateCode { footer-element } { talk } { 1 }
  {
    color      = \l_@@_footelem_color_tl ,
    font       = \l_@@_footelem_font_tl ,
    left-skip  = \l_@@_footelem_left_skip ,
    right-skip = \l_@@_footelem_right_skip
  }
  {
    \tl_if_empty:nF {#1}
      {
        \hspace { \l_@@_footelem_left_skip }
        \group_begin:
          \tl_if_empty:NF \l_@@_footelem_color_tl
            { \color_select:V \l_@@_footelem_color_tl }
          \l_@@_footelem_font_tl
          #1
        \group_end:
        \hspace { \l_@@_footelem_right_skip }
      }
  }
\DeclareInstance { footer-element } { date } { talk } { }
\DeclareInstance { footer-element } { author } { talk } { }
\DeclareInstance { footer-element } { title } { talk } { }
\DeclareInstance { footer-element } { institute } { talk } { }
\DeclareInstance { footer-element } { framenumber } { talk } { }
%    \end{macrocode}
% \end{variable}
%
% \begin{variable}
%    {
%     \l_@@_header_bg_tl           ,
%     \l_@@_header_fg_tl           , 
%     \l_@@_header_font_tl         ,
%     \l_@@_header_ht_dim          ,
%     \l_@@_header_left_skip       ,
%     \l_@@_header_frametitle_bool ,
%     \l_@@_header_right_skip 
%   }
%   Templates for the header area. The background always covers the full width,
%   but the text area may be narrower. The setup here aims to avoid repeated
%   kerns but also dealing with complex conditionals, hence we always move to
%   the edge of the paper first then adjust as required.
%    \begin{macrocode}
\NewTemplateType { header } { 0 }
\DeclareTemplateInterface { header } { talk } { 0 }
  {
    background-color  : tokenlist,
    color             : tokenlist = structure ,
    font              : tokenlist = \normalfont ,
    height            : length = \Gm@tmargin + \headsep ,
    left-hspace       : skip = \Gm@lmargin ,
    print-frame-title : boolean = true ,
    right-hspace      : skip = \Gm@rmargin
  }
\DeclareTemplateCode { header } { talk } { 0 }
  {
    background-color  = \l_@@_header_bg_tl ,
    color             = \l_@@_header_fg_tl ,
    font              = \l_@@_header_font_tl ,
    height            = \l_@@_header_ht_dim ,
    left-hspace       = \l_@@_header_left_skip ,
    print-frame-title = \l_@@_header_frametitle_bool ,
    right-hspace      = \l_@@_header_right_skip
  }
  {
    \noindent
    \@@_wallpaper_hrule:Nnn
      \l_@@_header_bg_tl
      { \l_@@_header_ht_dim - \headsep }
      { \headsep }
    \skip_horizontal:n { \l_@@_header_left_skip }
    \group_begin:
      \tl_if_empty:NF \l_@@_header_fg_tl
        { \color_select:V \l_@@_header_fg_tl }
      \l_@@_header_font_tl
      \bool_if:NT \l_@@_header_frametitle_bool
        {
          \ExpandArgs { nnV }
            \UseInstance { frametitle } { header }
              \g_@@_frame_title_tl
        }
    \group_end:
  }
\DeclareInstance { header } { std } { talk } { }
\AddToHook { begindocument }
  {
    \DeclareInstanceCopy { header } { wallpaper } { std }
    \EditInstance { header } { wallpaper }
      { print-frame-title = false }
  }
%    \end{macrocode}
% \end{variable}
% \begin{variable}
%   {
%     \l_@@_footer_bg_tl       ,
%     \l_@@_footer_fg_tl       ,
%     \l_@@_footer_font_tl     ,
%     \l_@@_footer_order_clist ,
%     \l_@@_footer_sep_tl      ,
%     \l_@@_footer_font_tl     ,
%     \l_@@_footer_left_skip   ,
%     \l_@@_footer_right_skip 
%   }
%   Templates for the footer area. Again the margins are handled in stages:
%   here we do have a box for the content so the right skip is used, and we
%   avoid an overfull box by including consideration of the right margin of the
%   page layout.
%    \begin{macrocode}
\NewTemplateType { footer } { 0 }
\DeclareTemplateInterface { footer } { talk } { 0 }
  {
    background-color : tokenlist ,
    color            : tokenlist ,
    element-order    : commalist ,
    font             : tokenlist = \tiny ,
    left-skip        : length = \Gm@lmargin ,
    right-skip       : length = \Gm@rmargin ,
    separator        : tokenlist = \hfil
  }
\DeclareTemplateCode { footer } { talk } { 0 }
  {
    background-color = \l_@@_footer_bg_tl ,
    color            = \l_@@_footer_fg_tl ,
    element-order    = \l_@@_footer_order_clist ,
    separator        = \l_@@_footer_sep_tl ,
    font             = \l_@@_footer_font_tl ,
    left-skip        = \l_@@_footer_left_skip ,
    right-skip       = \l_@@_footer_right_skip
  }
  {
    \noindent
    \@@_wallpaper_hrule:Nnn
      \l_@@_footer_bg_tl
      { \footskip }
      { \Gm@bmargin - \footskip }
    \skip_horizontal:n { \l_@@_footer_left_skip }
    \vbox_set_to_wd:Nnn \l_@@_tmp_box
      {
          \paperwidth
        - \l_@@_footer_left_skip
        - \l_@@_footer_right_skip
      }
      {
        \tl_if_empty:NF \l_@@_footer_fg_tl
          { \color_select:V \l_@@_footer_fg_tl }
        \l_@@_footer_font_tl
        \clist_pop:NNT \l_@@_footer_order_clist \l_@@_tmp_tl
          {
            \ExpandArgs { nVv } \UseInstance { footer-element } \l_@@_tmp_tl
              { @ \l_@@_tmp_tl }
            \clist_map_inline:Nn \l_@@_footer_order_clist
              { 
                \l_@@_footer_sep_tl
                \ExpandArgs { nnv }
                  \UseInstance { footer-element } {##1} { @ ##1 }
              }
          }
        \hfil
      }
    \box_use_drop:N \l_@@_tmp_box
    \skip_horizontal:n { \l_@@_footer_right_skip - \Gm@rmargin }
  }
\DeclareInstance { footer } { std } { talk } { }
\AddToHook { begindocument }
  {
    \DeclareInstanceCopy { footer } { wallpaper } { std }
    \EditInstance { footer } { wallpaper }
      { element-order = }
  }
%    \end{macrocode}
% \end{variable}
%
% \begin{macro}{\@@_wallpaper_hrule:Nnn}
%   A simple abstraction for the top and bottom rules on the page.
%    \begin{macrocode}
\cs_new_protected:Npn \@@_wallpaper_hrule:Nnn #1#2#3
  {
    \skip_horizontal:n { -\Gm@lmargin }
    \tl_if_empty:NF #1
      {
        \group_begin:
          \color_select:V #1
          \rule:nnn { \paperwidth } {#2} {#3}
          \skip_horizontal:n { -\paperwidth }
        \group_end:
      }
  }
%    \end{macrocode}
% \end{macro}
%
% \begin{macro}{\ps@plain, \ps@wallpaper, \ps@talk}
%   Install a standard header and footer template, and redefine the
%   \texttt{plain} one as this will be used for frames without
%   \enquote{wallpaper} which still need core links, \foreign{etc}. We also
%   provide a version that only shows the visual elements: this is deliberately
%   using the same settings as the main templates.
%    \begin{macrocode}
\cs_set_nopar:Npn \ps@plain
  {
    \cs_set_nopar:Npn \@oddhead
      {
        \@@_section_tagged:
        \hfil
      }
    \cs_set_nopar:Npn \@oddfoot { }
    \cs_set_eq:NN \@evenhead \@oddhead
    \cs_set_eq:NN \@evenfoot \@oddfoot
  }
\cs_set_nopar:Npn \ps@wallpaper
  {
    \cs_set_nopar:Npn \@oddhead
      {
        \@@_section_tagged:
        \UseInstance { header } { wallpaper }
        \hfil
      }
    \cs_set_nopar:Npn \@oddfoot
      {
        \UseInstance { footer } { wallpaper }
        \hfil
      }
    \cs_set_eq:NN \@evenhead \@oddhead
    \cs_set_eq:NN \@evenfoot \@oddfoot
  }
\cs_new_nopar:Npn \ps@talk
  {
    \cs_set_nopar:Npn \@oddhead
      {
        \@@_section_tagged:
        \UseInstance { header } { std }
        \hfil
      }
    \cs_set_nopar:Npn \@oddfoot { \UseInstance { footer } { std } }
    \cs_set_eq:NN \@evenhead \@oddhead
    \cs_set_eq:NN \@evenfoot \@oddfoot
  }
\pagestyle { talk }
%    \end{macrocode}
% \end{macro}
%
% \subsection{The \texttt{frame} environment}
%
% \begin{variable}{\l_@@_frame_bool}
%  To track whether we are inside a frame or not.
%    \begin{macrocode}
\bool_new:N \l_@@_frame_bool
%    \end{macrocode}
% \end{variable}
%
% \begin{variable}{\g_@@_frame_tag_bool}
%   To track when a frame is being tagged: mainly needed for the header
%   (and as a result global).
%    \begin{macrocode}
\bool_new:N \g_@@_frame_tag_bool
%    \end{macrocode}
% \end{variable}
%
% \begin{variable}{\l_@@_frame_verb_bool}
%  Indicates that material was collected verbatim (and thus needs rescanning).
%    \begin{macrocode}
\bool_new:N \l_@@_frame_verb_bool
%    \end{macrocode}
% \end{variable}
%
% \begin{variable}{\g_@@_frame_int, \c@frame, \theframe, \@framenumber}
%   The overall frame number, including \LaTeX{} counter-like access.
%    \begin{macrocode}
\int_new:N \g_@@_frame_int
\cs_new_eq:NN \c@frame \g_@@_frame_int
\cs_new:Npn \theframe { \@arabic \c@frame }
\cs_new:Npn \@framenumber { \arabic { frame } }
%    \end{macrocode}
% \end{variable}
%
% The total frames can be handled using the kernel properties.
%    \begin{macrocode}
\property_new:nnnn { totalframes } { shipout } { -1 }
  { \int_use:N \g_@@_frame_int }
\AddToHook { enddocument / afterlastpage }
  { \property_record:nn { lastpage } { totalframes } }
%    \end{macrocode}
% 
% \begin{macro}{\@@_latexe_frame:n}
%   As we will need to re-define \cs{frame} but have it available inside
%   frames, a copy is made here.
%    \begin{macrocode}
\NewCommandCopy \@@_latexe_frame:n \frame
%    \end{macrocode}
% \end{macro}
%
% \begin{macro}{\@@_frame_process:nn}
%   Here, the frame content is received as the argument.
%    \begin{macrocode}
\cs_new_protected:Npn \@@_frame_process:nn #1#2
  {
    \int_gincr:N \g_@@_frame_int
    \bool_set_true:N \l_@@_frame_bool
    \@@_slide:nn {#1} {#2}
  }
%    \end{macrocode}
% \end{macro}
%
% \begin{macro}{\@@_frame_tag:n}
%   Wraps some content in tagging for a frame: we may have multiple of these
%   in one logical frame, but that is non-standard.
%    \begin{macrocode}
\cs_new_protected:Npn \@@_frame_tag:n #1
  {
    \tag_struct_begin:n { tag = frame }
    \int_gset:Nn \g_@@_frame_struct_int { \tag_get:n { struct_num } }
    \bool_gset_true:N \g_@@_frame_tag_bool
    #1
    \tag_struct_end:
  }
%    \end{macrocode}
% \end{macro}
% \begin{macro}{\@@_frame_notag:n}
%   The alternative: turn off tagging and suppress the function that would
%   tag the frame title.
%    \begin{macrocode}
\cs_new_protected:Npn \@@_frame_notag:n #1
  {
    \tag_mc_begin:n { artifact }
    \tag_suspend:n { frame }
    \bool_gset_false:N \g_@@_frame_tag_bool
    #1
    \par
    \tag_resume:n { frame }
    \tag_mc_end:
  }
%    \end{macrocode}
% \end{macro}
%
% \begin{macro}{frame, frame*}
%   The definition for the \env{frame} and \env{frame*} environments: the exact
%   interface at both the document and code levels is still open.
%    \begin{macrocode}
\bool_if:NTF \l_@@_frame_title_bool
  {
    \RenewDocumentEnvironment { frame }
      { D <> { all } = { action-spec } O { } +m +b }
      {
        \keys_set:nn { talk / frame } {#2}
        \bool_set_false:N \l_@@_frame_verb_bool
        \@@_frame_process:nn {#1} { \frametitle {#3} #4 }
      }
      { }
    \NewDocumentEnvironment { frame* }
      { D <> { all } = { action-spec } O { } +m c }
      {
        \keys_set:nn { talk / frame } {#2}
        \bool_set_true:N \l_@@_frame_verb_bool
        \tl_gset:Nn \g_@@_frame_title_tl {#3}
        \exp_args:Nne \@@_frame_process:nn {#1}
          { \tl_to_str:n { \frametitle } \exp_not:n { {#3} #4 } }
      }
      { }
  }
  {
    \RenewDocumentEnvironment { frame }
      { !D <> { all } = { action-spec } !O { } +b }
      {
        \keys_set:nn { talk / frame } {#2}
        \bool_set_false:N \l_@@_frame_verb_bool
        \@@_frame_process:nn {#1} {#3}
      }
      { }
    \NewDocumentEnvironment { frame* }
      { !D <> { all } = { action-spec } !O { } c }
      {
        \keys_set:nn { talk / frame } {#2}
        \bool_set_true:N \l_@@_frame_verb_bool
        \@@_frame_process:nn {#1} {#3}
      }
      { }
  }
%    \end{macrocode}
% \end{macro}
%
%    \begin{macrocode}
%</class>
%    \end{macrocode}
%
% \end{implementation}
%
% \PrintIndex
